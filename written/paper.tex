% Options for packages loaded elsewhere
\PassOptionsToPackage{unicode}{hyperref}
\PassOptionsToPackage{hyphens}{url}
%
\documentclass[
  12,
]{article}
\usepackage{lmodern}
\usepackage{amssymb,amsmath}
\usepackage{ifxetex,ifluatex}
\ifnum 0\ifxetex 1\fi\ifluatex 1\fi=0 % if pdftex
  \usepackage[T1]{fontenc}
  \usepackage[utf8]{inputenc}
  \usepackage{textcomp} % provide euro and other symbols
\else % if luatex or xetex
  \usepackage{unicode-math}
  \defaultfontfeatures{Scale=MatchLowercase}
  \defaultfontfeatures[\rmfamily]{Ligatures=TeX,Scale=1}
\fi
% Use upquote if available, for straight quotes in verbatim environments
\IfFileExists{upquote.sty}{\usepackage{upquote}}{}
\IfFileExists{microtype.sty}{% use microtype if available
  \usepackage[]{microtype}
  \UseMicrotypeSet[protrusion]{basicmath} % disable protrusion for tt fonts
}{}
\makeatletter
\@ifundefined{KOMAClassName}{% if non-KOMA class
  \IfFileExists{parskip.sty}{%
    \usepackage{parskip}
  }{% else
    \setlength{\parindent}{0pt}
    \setlength{\parskip}{6pt plus 2pt minus 1pt}}
}{% if KOMA class
  \KOMAoptions{parskip=half}}
\makeatother
\usepackage{xcolor}
\IfFileExists{xurl.sty}{\usepackage{xurl}}{} % add URL line breaks if available
\IfFileExists{bookmark.sty}{\usepackage{bookmark}}{\usepackage{hyperref}}
\hypersetup{
  pdftitle={Pricing Kernels and Risk Premia implied in Bitcoin Options},
  pdfauthor={Julian Winkel; Wolfgang K. Härdle},
  hidelinks,
  pdfcreator={LaTeX via pandoc}}
\urlstyle{same} % disable monospaced font for URLs
\usepackage[margin = 2.5cm]{geometry}
\usepackage{graphicx,grffile}
\makeatletter
\def\maxwidth{\ifdim\Gin@nat@width>\linewidth\linewidth\else\Gin@nat@width\fi}
\def\maxheight{\ifdim\Gin@nat@height>\textheight\textheight\else\Gin@nat@height\fi}
\makeatother
% Scale images if necessary, so that they will not overflow the page
% margins by default, and it is still possible to overwrite the defaults
% using explicit options in \includegraphics[width, height, ...]{}
\setkeys{Gin}{width=\maxwidth,height=\maxheight,keepaspectratio}
% Set default figure placement to htbp
\makeatletter
\def\fps@figure{htbp}
\makeatother
\setlength{\emergencystretch}{3em} % prevent overfull lines
\providecommand{\tightlist}{%
  \setlength{\itemsep}{0pt}\setlength{\parskip}{0pt}}
\setcounter{secnumdepth}{-\maxdimen} % remove section numbering
\usepackage{graphicx}
\usepackage{amsmath}
\usepackage{float}
\usepackage{hyperref}
\usepackage{setspace}
\usepackage{subfig}
\usepackage{caption}
\usepackage{eucal}
\floatplacement{figure}{H}

\title{Pricing Kernels and Risk Premia implied in Bitcoin Options}
\author{Julian Winkel\footnote{International Research Training Group 1792;
  Humboldt-Universität zu Berlin,
  \href{mailto:julian.winkel@hu-berlin.de}{\nolinkurl{julian.winkel@hu-berlin.de}}} \and Wolfgang K. Härdle\footnote{BRC Blockchain Research Center,
  Humboldt-Universität zu Berlin, Berlin, Germany; Sim Kee Boon
  Institute, Singapore Management University, Singapore; WISE Wang Yanan
  Institute for Studies in Economics, Xiamen University, Xiamen, China;
  Dept. Information Management and Finance, National Chiao Tung
  University, Hsinchu, Taiwan, ROC; Dept. Mathematics and Physics,
  Charles University, Prague, Czech Republic, Grants -- DFG IRTG 1792
  and the Czech Science Foundation's grant no. 19-28231X / CAS XDA
  23020303 gratefully acknowledged,
  \href{mailto:haerdle@hu-berlin.de}{\nolinkurl{haerdle@hu-berlin.de}}}}
\date{21 November 2022}

\begin{document}
\maketitle

\newcommand{\quantnet}{\hspace*{\fill} \raisebox{-1pt}{\includegraphics[scale=0.05]{/Users/julian/src/up/spd/written/logos/qletlogo_tr.png}}\,}
\newcommand{\quantinar}{\hspace*{\fill} \raisebox{-1pt}{\includegraphics[scale=0.05]{/Users/julian/src/up/spd/written/logos/quantinar_logo.png}}\,}

\definecolor{bittersweet}{rgb}{1.0, 0.44, 0.37} 
\definecolor{ao}{rgb}{0.0, 0.5, 0.0}
\definecolor{bananamania}{rgb}{0.98, 0.91, 0.71}
\definecolor{bluegray}{rgb}{0.4, 0.6, 0.8}

\pagenumbering{roman}

\begin{abstract}

\singlespacing 
Bitcoin Pricing Kernels (PK) are estimated using a novel data set from Deribit, the leading Bitcoin options exchange. The PKs, as the ratio between risk-neutral and physical density, dynamically reflect the change of investor preferences. Thus the PKs improve the understanding of investor expectations and risk premium in a new asset class. Bootstrap-based confidence bands are estimated in order to validate the results. Investors are heterogeneous in their risk profiles and preferences with respect to volatility and investment horizon. The empirical PKs turn out to be U-shaped for short-dated instruments and W-shaped for long-dated instruments. We find that investors are willing to pay a substantial risk premium to insure themselves against short-term price movements. The risk premium is smaller for longer-dated instruments and their traders are risk averse. The shape of the empirical PKs reveals the existence of a time-varying risk-premium. The similarity between the shape of empirical PKs for Bitcoin and other markets that represent aggregate wealth shows that Bitcoin is becoming an established asset class. 


 \medskip
 \textbf{Keywords:} Bitcoin, Deribit, pricing kernel, risk aversion, speculation, hedging  \\
 \indent \textbf{JEL Classification:} C14, C50, G10  \\
 \bigskip

\end{abstract}

\pagenumbering{arabic}

\newpage

\section{Introduction}

The valuation of digital currencies has been a question that predated
its first representative, Bitcoin. Upon the creation of Bit Gold, Szabo
(2008) envisioned its value to arise from an interplay between the
benefits of the two entities which foremost represent value: cash and
metal. The argument put forward was that metal bears an inherent value
that is largely independent of trusted third parties. However, the
unwieldiness of metal incurs a relatively high transaction cost when
used as a means of exchange. (Unbacked) Cash, on the other hand, draws
its value largely from the trust in a third party's acceptance. The
combination of the two elements, coming in the form of a scarce digital
resource that is governed in a trustless system, has been passed on to
Bitcoin (BTC) and other Cryptocurrencies (CC).

Since the inception of BTC as proposed by Nakamoto (2009), various
arguments have been made on its valuation. Apart from the benefit of a
trustless system, the arguments typically include the production cost of
a block (Hayes 2019), the benefit of borderless transactions (Deng 2020;
Stellar Development Foundation 2022), the value of an inflation hedge
(Choi and Shin 2022), network effects (Chen and Vinogradov 2021; Gandal
and Halaburda 2016) or expectations about future price developments
(Blau 2017; Smaniotto and Neto 2022).

In addition to the former valuation approaches, we offer new insights by
means of the BTC derivatives market which has emerged over the course of
the last years. Since derivatives markets are particularly rich in
information, their evolution provides a unique opportunity to assess the
BTC market valuation through the application of proven econometric
techniques. Key information about preferences and forward looking
decisions are state price densities (SPD), that can be estimated from
option prices. SPDs yield risk-neutral probabilities, under which
investors price derivatives. They provide the key to pricing exotic or
illiquid options in an arbitrage-free manner (Aït-Sahalia and Lo 1998)
and offer insight into changing expectations about future developments.
In conjunction with the physical density (PD) of Bitcoin returns, the
resulting pricing kernels (PK) can be calculated. The shape and
evolution of PKs over time discloses investor expectations under
different market circumstances and gives insight into time varying
preferences. We present the first paper that is based on real Bitcoin
option data, inferring and disclosing investor preferences following
hypothesizing papers on bootstrap-based confidence bands for empirical
PKs (Härdle et al. 2014) and cryptocurrency option pricing under an SVCJ
model (Hou et al. 2020).

BTC derivatives markets can already be regarded as efficient information
processing mechanisms (Alexander et al. 2022). Among those markets,
\href{https://www.deribit.com/}{Deribit} is the leading crypto option
exchange as measured by open interest and trading volume. As of
2022-07-25, Deribit manages more than 90\% of the BTC option volume
which translates over a 30 day window into an average 24-hour trading
volume of over 331.8 million USD (Skew.com 2022). The competitors
LedgerX, OKEx, CME, bit.com and Binance are contributing respective
averages of 1.9, 15.7, 13.2, 1.5, 0.582 million USD.

A particularly interesting property of the BTC options market is its
decentralized nature. Trading on Deribit is largely dominated by
retailers. Despite an ongoing decline of retail market share in favor of
institutions (Coinbase 2022), who are cautious to invest in highly
volatile and unregulated assets, the options market on Deribit remains
driven by retailers: As of 2022-07-25, 86.65\% of the volume is
attributed to retail, whereas 9.16\% and 1.19\% are attributed to
investors and whales (Deribit 2022a). Deribit classifies market
participants according to their share of the circulating BTC supply.
According to Deribit's classification, a ``whale'' is an entity that
owns more than 1\% of the supply on Deribit, an ``investor'' owns
between 0.1\% and 1\% and a ``retailer'' owns less than 0.1\%.
Considering the retail share in trading volume in conjunction with
Deribit's dominating market share over the competition, we figure that
the Bitcoin price is mainly driven by retail. Analyzing a retail-driven
marketplace renders the study of digital currencies like BTC unique and
different from the well studied equity options markets, where retail
only has a minor influence (Bloomberg 2021).

Another interesting feature of the market under consideration is not
just the dominating presence of retailers, but their easy access to
leverage for speculation. Since Deribit is by design a margin-trading
platform, levers of up to 100 are available for longs. Easy access to
leverage could suggest the existence of the leverage effect if investors
were risk affine. However, we cannot confrm a leverage effect to be
present in the time frame under consideration.

All underlying code is available on \href{quantlet.com}{quantlet.com}
and a corresponding courselet is available on
\href{quantinar.com}{quantinar.com}.

\section{Pricing Kernels}

Assume a risky asset with a stochastic price process
\(\{S_t\}_{t \in \mathbb{N}}\) and a risk-free interest rate
\(\{r_t\}_{t \in \mathbb{N}}\) in a complete market. Following the
second Fundamental Theorem of Asset Pricing, a unique
martingale-equivalent measure \(Q\) exists in the described setting,
under which derivatives can be priced in an arbitrage-free manner (Huynh
et al. 2002; Pascucci and Agliardi 2011).

Let \(C_t\) be the price at time \(t\) of a contingent claim with payoff
\(\psi(\cdot)\) on the risky asset (underlying), which has a maturity at
\(T\) and a time-to-maturity \(\tau = T-t\). For simplicity assume a
constant interest rate \(r_t = r\). The price of any such contingent
claim can be expressed as the discounted value of expected future
payoffs, weighted with their respective probabilities of occurrence. The
expectation operator is conditional on the information set at \(t\)
under the equivalent martingale probability \(Q\)

\begin{equation}
C_t = e^{-r \tau} \mathbb{\sf{E}}^Q[\psi(S_T)] = e^{-r, \tau}  \int_{-\infty}^\infty \psi(u) f_t^Q(u) du
\end{equation}

Transforming the risk-neutral measure \(Q\) to the physical measure
\(P\) yields the PK by Itô's Lemma.

\begin{equation}
C_t = e^{-r, \tau}  \int_{-\infty}^\infty \psi(u) q(u) du = e^{-r, \tau}  \int_{-\infty}^\infty \psi(u) p(u) K(u) du
\end{equation}

where the PK \(K(\cdot)\) is defined as \(\frac{q(\cdot)}{p(\cdot)}\)
and for simplicity of notation we write \(q(\cdot)\) for
\(\frac{\partial Q}{\partial t}(\cdot)\) and drop the sequence \(r_t\).

The PK can therefore be approximated by the ratio between the
risk-neutral density and the physical density. This process is discussed
and executed in the following sections.

\section{Nonparametric Estimation of State Price Densities}

\subsection{Derivation}

As stated by Breeden and Litzenberger (1978), a SPD can be estimated via
the second derivative of the call price function
\(\psi(S_T) = \max(S_T - K, 0)\) with respect to the strike price \(K\).

\begin{equation}
\frac{\partial ^2 C_t}{\partial K^2}\bigg\rvert_{K = S_t} =  q(S_T)e^{-r \tau}
\end{equation}

A variety of call prices \(C\) with different strikes \(K\) is required
in order to calculate the complete SPD \(q\). The present value of a
call can be priced in implied volatility (IV). In conjunction with the
vector (time-to-maturity \(\tau\), strike \(K\), spot \(S\), interest
rate \(r\)), the market call price can be calculated.

IV is estimated as a function of time-to-maturity and moneyness in the
following section. Collapsing spot price \(S\) and strike \(K\) into a
single variable \(M = \frac{K}{S}\) reduces the effect of the curse of
dimensionality. Similarly, we collapse \(S\) and the interest rate \(r\)
into a Futures price.

\subsection{Local Polynomial Estimation of the IV Surface}

Assume that the implied volatilites have some noise added (Huynh et al.
2002; Rookley 1997).

\begin{equation}
\sigma (M, \tau) = g(M, \tau) + \sigma^{*}(M, \tau)  \varepsilon
\end{equation}

with a standardized error random variable \(\varepsilon\), moneyness
\(M\), \(\tau\) and \(\varepsilon\) independent and
\(\sigma^{*}(M, \tau)\) being the scaling of the error term given \(M\)
and \(\tau\). Suppose \(g\) is smooth, i.e.~it can be approximated using
Taylor's Theorem.

Taylor expansion of \(g\) in a neighborhood of \((M_0, \tau_0)\):
\begin{equation}
\begin{aligned}
g(M, \tau) = g(M_0, \tau_0) +
\frac{\partial g}{\partial M}(M-M_0) + 
\frac{1}{2} \frac{\partial^2 g}{\partial M^2}(M-M_0)^2 + \\
\frac{\partial g}{\partial \tau}(\tau - \tau_0) + 
\frac{1}{2} \frac{\partial^2 g}{\partial \tau^2}(\tau - \tau_0)^2 + \\ 
\frac{\partial^2 g}{\partial M \partial \tau}(M-M_0)(\tau - \tau_0)
\end{aligned}
\end{equation}

The functional relationship between the IV surface \(\sigma\) and \(M\)
and \(\tau\) can now be approximated using a Weighted Least Squares
Estimator (WLSE), minimizing the objective function

\begin{equation}
\operatorname{arg}\, \underset{\beta}{\operatorname{min}}{(\sigma - X\beta)^\top W (\sigma - X\beta)}
\end{equation}

where
\(W = \text{diag}(\mathcal{K}_{h_m, h_\tau}(M_j - M_0, \tau_j - \tau_0))\)
for a Gaussian kernel \(\mathcal{K}\) with bandwidths \(h_m\) and
\(h_\tau\). \(\sigma\) is the \(n \times 1\) vector of observed implied
volatilities, \(\beta\) is the vector of the local polynomial
coefficients.

\[
X = 
\begin{pmatrix}
1 & (M_1 - M_0) & (M_1 - M_0)^2 & (\tau_1 - \tau_0) & (\tau_1 - \tau_0)^2 & (M_1 - M_0) (\tau_1 - \tau_0) \\
\vdots & \vdots & \vdots & \vdots & \vdots & \vdots & \\
1 & (M_n - M_0) & (M_n - M_0)^2 & (\tau_n - \tau_0) & (\tau_n - \tau_0)^2 & (M_n - M_0) (\tau_n - \tau_0) \\
\end{pmatrix}
\]

The resulting WLSE is

\begin{equation}
\hat{\beta} = (X ^\top WX)^{-1}X ^\top W \sigma
\end{equation}

Following Härdle et al. (2014), a window of the last 500 daily returns
(based on \(\tau\)) is used to calculate a nonparametric Kernel Density
Estimator for the PD.\\
\href{https://quantinar.ro/course/103/statistics-of-financial-markets}{\hspace*{\fill} \raisebox{-1pt}{\includegraphics[scale=0.05]{/Users/julian/src/up/spd/written/logos/quantinar_logo.png}}\,Chapter 24}

\section{Literature Review}

Breeden and Litzenberger (1978) derive SPDs using Arrow-Debreu prices
and Butterfly Spreads. Their paper was the cornerstone for the now
existing vast literature on estimation of SPDs. Without requiring a
parametric form for SPDs, Rookley (1997) who developed a nonparametric
estimation method. IV skews are estimated by decomposing the functional
relationship between IV, moneyness and time-to-maturity \(\tau\). In
this manner it is possible to derive the SPD at every point in a robust
way. Aı̈t-Sahalia et al. (2001) estimate PKs from S\&P500 options data
and the according return series in order to assess the efficiency of the
options market. Departures from SPD and PD are used to identify
inefficient pricing. They design a trading strategy, exploiting the
skewness and kurtosis of the densities. The strategy is shown to have a
high Sharpe Ratio. Grith et al. (2009) propose a systematic modeling
approach to study the evolution of PKs over time. With European DAX
data, a series of empirical PKs is estimated from 2003 until 2006. While
the risk-neutral density is inferred using Rookley's method, the PD is
estimated with a GARCH model. A common shape is identified and
deviations of the time varying EPKs are studied. The deviations between
individual PKs are described using a set of parameters for horizontal
and vertical shifts. The relationship between PKs and Arrow-Pratt
measure of Absolute Risk Aversion (ARA) give insight into investor's
risk aversion. In a related research, Härdle et al. (2014) derive
bootstrap-based confidence bands for nonparametrically estimated PKs.

Inverse Options, meaning options settled in kind of the underlying, are
dominant in the crypto world. Settlement of BTC options in terms of BTC
changes the payoff function of the option from \(\max(S_T - K; 0)\) to
\(\max(\frac{S_T - K}{S_T}; 0)\) and thus changes the contract's net
present value. Alexander and Imeraj (2021) adjust Black-Scholes prices
and hedge ratios to inverse options. However, it is argued that traders
are erroneously applying vanilla Black-Scholes valuation instead of the
corrected prices - perhaps because of being unaware of the concept. We
have decided to use with the standard, non-inverse Black Scholes pricing
for multiple reasons. Alexander and Imeraj (2021) argue that traders are
perhaps unaware of the difference. Since we are inferring investor
expectations, we figure that it is more appropriate to infer pricing
kernels under the same Black Scholes prices that the investors use.
Additionally, it is likely that many traders actually hedge their
overall Bitcoin exposure. This would negate the ``inverse'' part of the
option. Furthermore, reviewing the differences in pricing compared to
the adjusted prices, we find that the difference is small in the absence
of extreme moves in the underlying. This was not the case in the time
period under consideration.

Hou et al. (2020) price BTC options under an SVCJ model. Their results
emphasize the tail risk introduced by jumps in the underlying. Jumps in
particular introduce market incompleteness, but anonymity of
transactions may be relevant as well. Chen and Vinogradov (2021) derive
PKs and the impact of market incompleteness on risk premia. They state
that a key property of cryptocurrency valuation is the user's anonymity
(or pseudonymity), which is simultaneously the source of value and
incompleteness in the respective market. They argue that hedging an
anonymous transaction would require an identity disclosure. This
contradiction may introduce an effectively unhedgeable event.

\section{Deribit}

Deribit, a margin-trading platform for futures and options, has been
launched in June 2016 in the Netherlands and is currently incorporated
in Panama. As of time of the writing, Deribit is the largest BTC option
exchange (Skew.com 2022). For both types of derivatives, BTC and ETH are
the underlying as well as the currency in which settlement is conducted.
This allows to view them as inverse options.

\begin{center}
\begin{figure}
\includegraphics[width=0.9\textwidth]{/Users/julian/src/up/spd/plot/Skew_BTC_Open_Interest_cut.pdf}
\caption[Open Interest in BTC Options per Exchange.]%
{Open Interest in BTC Options per Exchange. Snapshot from \href{https://analytics.skew.com/dashboard/bitcoin-options}{skew.com} on 2022-01-16.  \color{bittersweet}{Deribit}, \color{ao}{LedgerX}, \color{bananamania}{CME}, \color{bluegray}{bit.com}. }
\label{fig: ivtest}
\end{figure}
\end{center}

\subsection{Bitcoin Option Contract Specifications}

Deribit offers cash-settled European-style options. The underlying is a
synthetic index, whose exact composition is described in the data
section. Settlement is first calculated in USD and then conducted in
kind of the underlying. E.g. a BTC call with a strike of \(10000\) and a
settlement value of \(12000\) at maturity will result in a cashflow of
\(2000\) USD, which is equivalent to \(\frac{1}{6}\) BTC (excluding
transaction cost). The settlement value is defined as the average of the
underlying BTC index for the last 30 minutes before settlement. Each
contract has a lot size of 1 BTC and is settled at 8am UTC on the
respective maturity date. Maturity dates have a daily, monthly,
quarterly and annual frequency. Deribit is the only derivatives exchange
to offer daily options. Instruments are trading 24/7 with a tick size of
0.0005 BTC. Due to the automatic usage of margin-trading, margin-based
liquidation of positions is possible. Initial margin and maintenance
margins are both zero for long positions. For short positions, initial
margin is required (Deribit 2022b). In case of liquidation, a penalty is
applied to the defaulting party, whose proceeds are paid into the
Deribit Insurance Fund (Deribit 2020a).

Trading and deliveries on Deribit are subject to fees (Deribit 2022c).
The applied fees generally vary depending on whether the order was a
liquidity maker or taker. However, for Bitcoin options they are equal.
Trading fees are 0.03\% of the underlying or 0.0003 BTC per options
contract. The fees constitute a maximum of 12.5\% of the contract's
value.

\section{Data}

\subsection{Data Structure}

The dataset consists of 8,444,664 order book snapshots, which have been
collected from Deribit in the time from 2021-04-01 until 2022-04-01. The
snapshots were captured via the Deribit API V2. The data base is
available on the
\href{https://blockchain-research-center.com}{Blockchain Research Center (BRC)}.
The full data set includes all parameters that the
\href{https://docs.deribit.com/#deribit-api-v2-1-1}{Deribit API V2}
returns at the time of collection under the methods

\begin{itemize}
\tightlist
\item
  \{\it public/get\_last\_trades\_by\_instrument\_and\_time\}
\item
  \{\it public/get\_order\_book\}
\end{itemize}

Most importantly the results include

\begin{itemize}
\tightlist
\item
  Timestamps
\item
  Greeks
\item
  Implied Volatility
\item
  Tick Direction
\item
  Order Type
\item
  Volume
\item
  Instrument Price
\item
  Strike
\item
  Spot
\end{itemize}

High-frequent order book changes and executed trades are captured for
options and futures whose underlying is BTC. Values of the underlying
synthetic BTC USD Index are also saved. The underlying is calculated as
an equally-weighted BTC/USD price of eleven major crypto exchanges,
namely Binance, Bitfinex, Bitstamp, Bittrex, Coinbase Pro, FTX, Gemini,
Huobi, Itbit, Kraken, LMAX Digital, OKEx. Individual feeds can be
excluded due to administrative decisions or in case if invalid data.
Remaining feeds are sorted, truncated around the median price and
weighted (Deribit 2020b).

\subsection{Data Integrity}

Data integrity plays a crucial role in crypto markets. Mark Carney, the
chair of the Financial Stability Board and head of the Bank of England,
warned that wash trading, pump and dumps and spoofing, known as outlawed
manipulation techniques in equity markets, are also present in crypto
and pose a risk to financial stability (Rodgers 2019). Wash trading is
used to increase trading volume and thus artificially increase demand. A
spoofer submits market non-bona fide price quotes in order to cause
artificial price volatility (Sar 2017). Pump \& Dump is a form of
securities fraud in which a group of traders rapidly and artificially
inflate a price in order to offload their previously acquired inventory.
Since Pump \& Dump schemes require low liquidity (in the underlying), we
can exclude this possibility for Bitcoin (La Morgia et al. 2020).

Aloosh and Li (2019) find evidence for exchange-driven inflated volume,
generated by wash trades, to market themselves under the guise of
liquidity. Cong et al. (2019) report that on average 70\% of volume on
decentralized exchanges is fake due to the use of wash trading.
According to Bitmex (2019), up to 95\% of trading volume on unregulated
exchanges is generated by wash trading. In order to ensure data
integrity, we exclusively use orderbooks instead of executed trades in
order to filter possible cases of wash trading.

To address the potential issue of spoofing, we adopt an approach related
to Tuccella et al. (2021). They attempt to identify spoofing on
cryptocurrency exchanges using a GRU model. Their predictors are a
function of the amount of cancelled orders relative to the cumulative
order size on each side of the order book. Since Bitcoin derivatives
order books do not have the same amount of depth as spot or futures
markets, we employ a stricter method and exclude all orders whose
lifetime does not exceed two seconds. We find that less than 5\% of
orders are filtered due to short order lifetime. We figure that
traditional spoofing techniques, based on the rapid submission and
cancellation of orders, are difficult to implement due to Deribit's rate
limits (Deribit 2023). Users with less than 1 million USD in
7-day-turnover can only post five matching engine requests. These
include request for a buy or sell order or a corresponding cancellation.
Although the limits for a market maker (who could be allowed to post 30
requests per second given a turnover of 25 million USD in 7 days) are
considerably higher, the market maker must still meet the quoting
requirements for a large range of instruments. This effectively
restricts the influence that single entities can have on certain
instruments via the rapid submission and cancellation of orders and
explains the rarity of such events in our dataset.

\subsection{Preprocessing}

Although market makers are obliged to quote most instruments for the
majority of the time, some may not be quoted at all or only be quoted at
the cost of a large spread (Deribit 2020c). For Far-Out-of-Money (FOTM)
contracts, when the minimum tick size exceeds the market IV, one can
observe bids for 0\% IV. Cases of quotes for 0\% IV and all duplicates
are excluded from the data set. Since the majority of option trading
volume concentrates on instruments with short or medium time-to-maturity
\(\tau\), we restrict \(\tau\) to be smaller or equal than one quarter
of a year. \(\tau\) is normalized on the span of a year. Instruments are
grouped according to their time-to-maturity, over a range of 0, 1, 2, 4,
8 weeks. We restrict moneyness \(M = \frac{K}{S}\) to the interval
\([0.7, 1.3]\) in order to exclude the influence of high-volatility
observations which could be unreliable (Grith et al. 2009).

Call options are exclusively used in order to estimate the IV surface in
Rookley's method. Put-Call-Parity ensures arbitrage-free call option
prices. However, we do not use Put-Call-Parity to increase our data
sample. Conversion of option prices via Put-Call-Parity could introduce
errors into our dataset due to market microstructure (MMN) noise. Our
high-frequent order book data is extracted via an API. Due to rate
limitations and a diverse amount of instruments, order book snapshots
cannot be taken simultaneously for all instruments. The data scraping
application successively iterates over all instruments and captures
changes in order books. While the resulting data base provides a clear
picture of order book data, it does not allow for a complete
reconstruction of all order books at all times. Without simultaneity,
volatility of the underlying and order competition within the spread
could change the prices of the corresponding instruments (required for
Put-Call-Parity) between the individual snapshots. A range of futures
may be available on Deribit, but only the perpetual futures contract is
actually liquid. The classic (non-perpetual) futures also price in the
settlement cost, which often causes them to trade in backwardation close
to maturity. But in spite of this, BTC options tend to be traded more
frequently when they are close to maturity. The combination of these
real-world limitations challenges the simple conversion via
Put-Call-Parity and could introduce errors into our dataset.
Furthermore, we do not see the necessity to use such a method as we have
sufficient call option data to compute the IV surface from which we
sample option prices.

\href{https://github.com/QuantLet/BitcoinOptions/blob/master/main.py}{\hspace*{\fill} \raisebox{-1pt}{\includegraphics[scale=0.05]{/Users/julian/src/up/spd/written/logos/qletlogo_tr.png}}\,Preprocessing}

\section{Volatility}

In accordance with Masset (2011) and Eraker (2021), the list of
well-established stylised facts on volatility in traditional markets
include horizontal dependence of volatility, leverage effect, the
volatility premium and extreme events. Horizontal dependence of
volatility describes the tendency of local volatility clusters and the
tendence to mean-reversion. Both properties can be observed in the 7 day
rolling volatility. Sudden spikes in realized volatility form clusters
and are often followed by similar drops. The leverage effect can be
measured as a negative correlation between returns and volatility.
Phases of negative returns coincide with high volatility and vice versa.
The Pearson correlation between the underlying and realized volatility
is -12.59\%. The Pearson correlation between the underlying and the bid
(ask) IV is -16.11\% (-13.05\%). Due to the low absolute values of
correlation, the existence of a leverage effect is unlikely. However,
the sign matches the direction of a leverage effect. Considering the
large share of retailers with easy access to leverage, the lack of
significance is a surprising result. It is well known that, in general,
the IV exceeds the unconditional annualized standard deviation. This is
displayed in Figure \ref{fig: iv_comparison}. The difference between the
implied and realized volatility is commonly referred to as volatility
premium. The mean 7 day rolling volatility is 70.96\%, whereas the mean
IV at the bid and ask are 85.28\% and 93.57\%. Spikes in realized
volatility rarely exceed the ask of the IV. Consequently, the volatility
premium is shown to be substantial throughout the regarded time period.
Fat-tail events are priced in due to the classic volatility skew. This
is depicted in Figure \ref{fig: emp_ivsmile}, the empirical volatility
skew.

Since the data are high frequency order book snapshots with substantial
bid-ask spreads, we must address the issue of Market Microstructure
Noise (MMN). A simple estimate of realized volatility would increase
with our sample due to order competition within the spread. We resolve
the issue of MMN in our calculation of realized volatility by taking the
arithmetic mean after aggregating the implied volatility of bids and
asks on a daily base. Therefore the frequency is matching the underlying
BTC index. The formula for annualized, realized volatility of price
\(p\) over a window of \(w\) days is

\begin{equation}
\sigma_w = \sqrt{\frac{1}{w}\sum_{t=1}^w{(\log{p_t} - \log{p_{t-1})}^2}} \sqrt{\frac{365}{w}}
\end{equation}

Annualization is based on 365 days since Bitcoin options are traded
continuously on Deribit.

The term structure itself is easier to assess when regarding ATM-options
in a series of dates (Figure 6). ATM-options are defined as those
options, whose moneyness falls in the interval \([0.9, 1.1]\). A typical
term structure of the Bitcoin options reveals a high level of implied
volatility for shorted-dated options. For options with a
time-to-maturity of three days or more, implied volatility drops sharply
compared to the initial level. Naturally, IV is increasing with
time-to-maturity.

\begin{center}
\begin{figure}
\includegraphics[width=1\textwidth]{/Users/julian/src/up/spd/plot/iv_comparison_up.png}
\caption[Implied Volatiliy vs. Realized Volatility on Deribit]%
{Implied volatility vs. realized volatility. Realized volatility is annualized and regarded in a 7 day window.
IV is calculated as the average of observed orderbooks at the bid and ask. Described data filters apply. {\color{blue} Ask IV}, {\color{green} Bid IV}, {\color{red} 7 Day Rolling Volatility}.}
\label{fig: iv_comparison}
\href{https://github.com/QuantLet/BitcoinOptions/blob/master/src/vola.py}{\hspace*{\fill} \raisebox{-1pt}{\includegraphics[scale=0.05]{/Users/julian/src/up/spd/written/logos/qletlogo_tr.png}}\,Volatility}
\end{figure}
\end{center}

\begin{centering}
\begin{figure}
\includegraphics[width=1\textwidth]{/Users/julian/src/up/spd/plot/vola_2021-05-23.png}
\caption[Empirical Volatility Surface]{Volatility Skew shown in the empirical Volatility Surface on 2021-05-23}
\label{fig: emp_ivsmile}
\end{figure}
\end{centering}

\begin{centering}
\begin{figure}
\includegraphics[width=1\textwidth]{/Users/julian/src/up/spd/plot/vola_2021-09-08.png}
\caption[Empirical Volatility Surface 2]{Volatility Skew shown in a calmer period on 2021-09-08}
\label{fig: emp_ivsmile2}
\href{https://github.com/QuantLet/BitcoinOptions/blob/master/src/vola_plots.py}{\hspace*{\fill} \raisebox{-1pt}{\includegraphics[scale=0.05]{/Users/julian/src/up/spd/written/logos/qletlogo_tr.png}}\,IV Smile}
\end{figure}
\end{centering}

\begin{minipage}{.5\textwidth}
    \centering
    \includegraphics[width=2.8in]{/Users/julian/src/up/spd/plot/term_structure_atm_options_on_-2022-01-010.08.png}
    \centering
    \includegraphics[width=2.8in]{/Users/julian/src/up/spd/plot/term_structure_atm_options_on_-2022-01-020.08.png}
    \centering
    \includegraphics[width=2.8in]{/Users/julian/src/up/spd/plot/term_structure_atm_options_on_-2022-01-030.08.png}
    \centering
    \includegraphics[width=2.8in]{/Users/julian/src/up/spd/plot/term_structure_atm_options_on_-2022-01-040.08.png}

\end{minipage}
\begin{minipage}{0.5\textwidth}
    \centering
    \includegraphics[width=2.8in]{/Users/julian/src/up/spd/plot/term_structure_atm_options_on_-2022-01-050.08.png}
    \centering
    \includegraphics[width=2.8in]{/Users/julian/src/up/spd/plot/term_structure_atm_options_on_-2022-01-060.08.png}
    \centering
    \includegraphics[width=2.8in]{/Users/julian/src/up/spd/plot/term_structure_atm_options_on_-2022-01-070.08.png}
        \centering
    \includegraphics[width=2.8in]{/Users/julian/src/up/spd/plot/term_structure_atm_options_on_-2022-01-080.08.png}

\end{minipage}
    \captionof{figure}{ATM Term Structure from 2022-01-01 until 2022-01-08. Moneyness $M \in [0.9, 1.1]$}

\label{ATM Term Structure}
\href{https://github.com/QuantLet/BitcoinOptions/blob/master/main.py}{\hspace*{\fill} \raisebox{-1pt}{\includegraphics[scale=0.05]{/Users/julian/src/up/spd/written/logos/qletlogo_tr.png}}\,Main}

\pagebreak
\newpage
\section{Empirical Pricing Kernels}

Instruments are grouped by time-to-maturity in order to summarize the
findings. All instruments with less than one week to maturity are
classified as having less than one, between one and two, between two and
three, between three and four and between five and eight weeks to
maturity. We typically observe U-shaped PKs for short-term maturities.
In a similar fashion to the oil market, investors perceive short-term
price changes in any direction as undesirable and prefer to hedge their
risk (Christoffersen et al. 2021). For short-dated instruemnts, option
writers are asking for substantial risk premia in order to reflect
unhedgeable risks, such as jumps in the underlying (Hou et al. 2020).

We find a differnet shape for longer-dated instruments. Typically, PKs
in traditional markets are monotonically decreasing. This results from
human risk aversion; investors are willing to insure themselves against
losses and have a preference for smooth consumption curves. However, PKs
outside of index options markets are not well studied (Cuesdeanu and
Jackwerth 2018). Grith et al. (2009) compartmentalize a pricing kernel
in two segments around a possible breakpoint. Testing an unrestricted
model against GMM estimates of the restricted one, they reject
monotonicity of the empirical PK in four out of five cases (performing a
D-test). The phenomenon of having a decreasing slope, although with
locally increasing sections in empirical PKs is called the pricing
kernel puzzle. We observe a similar behavior and extend the existing
literature on the pricing kernel puzzle to a new asset class.

Since the empirical PKs for short-dated instruments are U-shaped, we
figure that investors are willing to pay a high risk premium in order to
insure immediate price risks. With an increasing maturity of the
instrument, the shape of the empirical PK resembles a ``W'' or
``tilde''. This corresponds to a a lesser willingness to pay a risk
premium.\\
The dynamic of the empirical PKs also reveals the existence of a
time-varying risk premium in BTC markets. The observed PK shapes are
closely aligned with the results reported in Cuesdeanu and Jackwerth
(2018). However, their object of study are traditional markets, such as
major indices which are highly correlated with aggregate wealth. We
conclude that, due to the similarities of the empirical PKs, BTC is well
on its way to becoming an established asset class. It may prove to be
much rather a store of value than it is currently given credit for.

Apart from the discussion of results, deviations between PD and SPD can
be interpreted as trading opportunities. (Blaskowitz et al. 2004) design
and evaluate trading strategies based on deviations between PD and SPD.
Estimated PKs could be used by traders employing so-called Skewness or
Kurtosis trades. Nevertheless, traders must recognize the hidden cost of
employing such strategies in terms of substantial hedging cost, e.g.~in
the form of volatility and large spreads.

\begin{centering}
\begin{figure}
\includegraphics[width=1\textwidth]{/Users/julian/src/up/spd/plot/Pricing Kernel_over_time_nweeks=0.png}
\caption[Empirical Pricing Kernel - 0 Weeks to Expiration]{Pricing Kernels with less than one Week to Expiration}
\end{figure}
\end{centering}

\begin{centering}
\begin{figure}
\includegraphics[width=1\textwidth]{/Users/julian/src/up/spd/plot/Pricing Kernel_over_time_nweeks=1.png}
\caption[Empirical Pricing Kernel - 1 Week to Expiration]{Empirical Pricing Kernel - 1 Week to Expiration}
\end{figure}
\end{centering}

\begin{centering}
\begin{figure}
\includegraphics[width=1\textwidth]{/Users/julian/src/up/spd/plot/Pricing Kernel_over_time_nweeks=2.png}
\caption[Empirical Pricing Kernel - 2 Weeks to Expiration]{Empirical Pricing Kernel - 2 Weeks to Expiration}
\end{figure}
\end{centering}

\begin{centering}
\begin{figure}
\includegraphics[width=1\textwidth]{/Users/julian/src/up/spd/plot/Pricing Kernel_over_time_nweeks=3.png}
\caption[Empirical Pricing Kernel - 3 Weeks to Expiration]{Empirical Pricing Kernel - 3 Weeks to Expiration}
\end{figure}
\end{centering}

\begin{centering}
\begin{figure}
\includegraphics[width=1\textwidth]{/Users/julian/src/up/spd/plot/Pricing Kernel_over_time_nweeks=4.png}
\caption[Empirical Pricing Kernel - 4 Weeks to Expiration]{Empirical Pricing Kernel - 4 Weeks to Expiration}
\end{figure}
\end{centering}

\begin{centering}
\begin{figure}
\includegraphics[width=1\textwidth]{/Users/julian/src/up/spd/plot/Pricing Kernel_over_time_nweeks=8.png}
\caption[Empirical Pricing Kernel - 5-8 Weeks to Expiration]{Empirical Pricing Kernel - 5-8 Weeks to Expiration}
\href{https://github.com/QuantLet/BitcoinOptions/blob/master/main.py}{\hspace*{\fill} \raisebox{-1pt}{\includegraphics[scale=0.05]{/Users/julian/src/up/spd/written/logos/qletlogo_tr.png}}\,Main}
\end{figure}
\end{centering}

\pagebreak
\newpage

\section{Bootstrap-based Confidence Bands}

Having computed a large set of empirical PKs with varying maturities, it
should be ensured that the observed features are not mere artefacts. A
useful tool to assess the validity of our estimated PKs are
bootstrap-based confidence bands. Following the derivation of Härdle et
al. (2014), confidence bands at any confidence level can be derived via
the ``wild bootstrap method''. Figure
\ref{fig:Bootstrapped Confidence Bands - 2 Days} until Figure
\ref{fig:Bootstrapped Confidence Bands - 84 Days} are depicted
exemplarily for the whole set. Confidence bands are tighter around the
PKs when time to maturity increases. PKs are also flatter with growing
time to maturity.

The concept of bootstrapping PK can be introduced by linearization of
the PK. Stochastic deviation of the PK can be linearised into a
stochastic part, containing the estimator of the SPD and a deterministic
part containing the expectation of the physical density. Convergence of
both parts can be proven separately.

\[
\widehat{K}(x) = \frac{\widehat{q}(x)}{\widehat{p}(x)} 
\]

\[
\begin{aligned}
\underset{x \in E}{\operatorname{sup}} \lvert \widehat{K}(x) - K(x) \rvert = \underset{x \in E}{\operatorname{sup}} \lvert \frac{\widehat{q}(x) - q(x)}{p(x)} - \frac{\widehat{p}(x) - p(x)}{p(x)} * \frac{q(x)}{p(x)} - \frac{\{\widehat{q}(x) - q(x)\} \{\widehat{p}(x) - p(x)\} }{p^2(x)} \rvert  \\
+ {\scriptstyle \mathcal{O}_p}[\operatorname{max}((n_p h_{n_p}/\log)^{-1/2} + h^2_{np},h_{n_q}^{-2} \{ n_q h_{n_q} / \log{n_q} \}^{-1/2} + h^2_{n_q})]
\end{aligned}
\]

Consider the leading term of \[
\underset{x \in E}{\operatorname{sup}} \lvert \frac{\widehat{q}(x) - q(x)}{p(x)} \rvert
\]

Resample data from the smoothed bivariate distribution of strike and
moneyness (X,Y)

\[
{\widehat{f}(x,y)} = \frac{\widehat{\sigma_X}}{n_q h^2_{n_q}\widehat{\sigma_Y}} 
\Sigma_{i=1}^{n_q} K
\{ 
\frac{X_i - x}{h_{n_q}}, 
\frac{(Y_i - y)\widehat{\sigma_X}}{h_{n_q}\widehat{\sigma_Y}}
\}
\]

Using the resampled data, calculate the bootstrap analogue

\[
\underset{x \in E}{\operatorname{sup}} \lvert \frac{\widehat{q}^*(x) - \widehat{q}(x)}{p(x)} \rvert
\]

where
\(\underset{x \in E}{\operatorname{sup}} \lvert\widehat{q}^*(x) - \widehat{q}(x) \rvert\)
converges to the Gumbel distribution at an unfortunately slow rate of
\(\frac{1}{\log n_q}\) (Hall 1991).

\begin{minipage}{.5\textwidth}
    \centering
    \includegraphics[width=3.0in]{/Users/julian/src/up/spd/plot/EPK_2021-12-31_0.0055.pdf}
    \captionof{figure}{2 Days to Maturity.}
    \label{fig:Bootstrapped Confidence Bands - 2 Days}
    \centering
    \includegraphics[width=3.0in]{/Users/julian/src/up/spd/plot/EPK_2021-12-31_0.0192.pdf}
    \captionof{figure}{7 Days to Maturity.}
    \label{fig:Bootstrapped Confidence Bands - 7 Days}
    \centering
    \includegraphics[width=3.0in]{/Users/julian/src/up/spd/plot/EPK_2021-12-31_0.0384.pdf}
    \captionof{figure}{14 Days to Maturity.}
    \label{fig:Bootstrapped Confidence Bands - 14 Days}
    \centering
    \includegraphics[width=3.0in]{/Users/julian/src/up/spd/plot/EPK_2021-12-31_0.0575.pdf}
    \captionof{figure}{21 Days to Maturity.}
    \label{fig:Bootstrapped Confidence Bands - 21 Days}

\end{minipage}
\begin{minipage}{0.5\textwidth}
    \centering
    \includegraphics[width=3.0in]{/Users/julian/src/up/spd/plot/EPK_2021-12-31_0.0767.pdf}
    \captionof{figure}{28 Days to Maturity.}
    \label{fig:Bootstrapped Confidence Bands - 28 Days}
    \centering
    \includegraphics[width=3.0in]{/Users/julian/src/up/spd/plot/EPK_2021-12-31_0.1534.pdf}
        \captionof{figure}{56 Days to Maturity.}
    \label{fig:Bootstrapped Confidence Bands - 56 Days}
    \centering
    \includegraphics[width=3.0in]{/Users/julian/src/up/spd/plot/EPK_2021-12-31_0.2301.pdf}
        \captionof{figure}{84 Days to Maturity. \color{red}{Empirical Pricing Kernels} with \color{blue}{bootstrapped Confidence Bands} on 2021-12-31.}
    \href{https://github.com/QuantLet/BitcoinOptions/blob/master/main.py}{\hspace*{\fill} \raisebox{-1pt}{\includegraphics[scale=0.05]{/Users/julian/src/up/spd/written/logos/qletlogo_tr.png}}\,Main}
    \label{fig:Bootstrapped Confidence Bands - 84 Days}
\end{minipage}

\pagebreak
\newpage

\section{Conclusion}

Empirical PKs have been estimated using Rookley's Method on a dataset of
order book snapshots from Deribit, the leading BTC options market.
Bootstrap-based confidence bands have been estimated in order to
validate the results.

We assess the presence of well-established stylized facts of volatility
in the BTC market. We find horizontal dependence of volatility, meaning
that volatility tends to cluster and revert to the mean. The difference
between implied and realized volatility, commonly referred to as the
volatility premium, is substantial. Realized volatility rarely exceeds
the implied volatility. The IV term structure is found to be
traditional. Despite market participant's easy access to leverage, we
cannot confirm the existence of a leverage effect.

Our analysis extends the literature to a new asset class and contributes
to the pricing kernel puzzle. We are able to replicate empirical results
from the analysis of PKs in traditional markets, which are highly
correlated with aggregate wealth. Their similarity to PKs estimated from
BTC options indicates that BTC is becoming an established asset class.

Furthermore, our analysis sheds light on the BTC valuation and
risk-aversion of the retail traders. BTC option traders mostly consist
of retailers who are heterogeneous in their risk profile. The price of
short-dated instruments includes a high risk premium. Such instruments
reflect the anticipation of jumps in the underlying. These instruments
would be traded by either very risk-averse or risk-affine traders.
Long-dated instruments are employed as classic hedging instruments.
While a substantial risk premium is paid by traders in order to insure
themselves against falling prices, they are additionally (but to a
lesser degree) insuring themselves against sharply rising prices. Thus
investors are also hedging their risk of being priced out of a dynamic
market.

Several extensions of the presented research are available for the
future. Among those, we would find an extended dataset insightful, that
could support estimation of PKs and their evolution over a longer time
frame and possibly describe the convergence to the PKs of established
markets. It would also be interesting to analyze the evolution of PKs
around stress events in order to get further insights into investor risk
aversion, anticipation of such events and trading opportunities.

\section{References}

\hypertarget{refs}{}
\leavevmode\hypertarget{ref-aitsahalia01optionprice}{}%
Aı̈t-Sahalia, Y., Wang, Y., \& Yared, F. (2001). Do option markets
correctly price the probabilities of movement of the underlying asset?
\emph{Journal of Econometrics}, \emph{102}(1), 67--110.
\url{https://doi.org/https://doi.org/10.1016/S0304-4076(00)00091-9}

\leavevmode\hypertarget{ref-ait_sahalia_wang1998}{}%
Aït-Sahalia, Y., \& Lo, A. W. (1998). Nonparametric estimation of
state-price densities implicit in financial asset prices. \emph{The
Journal of Finance}, \emph{53}(2), 499--547.
\url{https://doi.org/https://doi.org/10.1111/0022-1082.215228}

\leavevmode\hypertarget{ref-alexander2022net}{}%
Alexander, C., Deng, J., Feng, J., \& Wan, H. (2022). Net buying
pressure and the information in bitcoin option trades. \emph{Journal of
Financial Markets}, 100764.

\leavevmode\hypertarget{ref-inverse_options}{}%
Alexander, C., \& Imeraj, A. (2021, July). Inverse options in a
black-scholes world.
\url{https://doi.org/https://doi.org/10.48550/arXiv.2107.12041}

\leavevmode\hypertarget{ref-aloosh}{}%
Aloosh, A., \& Li, J. (2019). Direct evidence of bitcoin wash trading.
\emph{Theory of Probability \& Its Applications, 5, 285-301}.
\url{https://doi.org/http://dx.doi.org/10.2139/ssrn.3362153}

\leavevmode\hypertarget{ref-bitmex}{}%
Bitmex. (2019). Presentation to the u.s. Securities and exchange
commission on wash trading.
\url{https://www.sec.gov/comments/sr-nysearca-2019-01/srnysearca201901-5164833-183434.pdf}.
Accessed 18 January 2022

\leavevmode\hypertarget{ref-blase}{}%
Blaskowitz, O., Härdle, W. K., \& Schmidt, P. (2004). Skewness and
kurtosis trades. In (pp. 1--14).
\url{https://doi.org/10.1007/978-0-8176-8180-7_1}

\leavevmode\hypertarget{ref-speculation}{}%
Blau, B. M. (2017). Price dynamics and speculative trading in bitcoin.
\emph{Research in International Business and Finance}, \emph{41},
493--499.
\url{https://doi.org/https://doi.org/10.1016/j.ribaf.2017.05.010}

\leavevmode\hypertarget{ref-bloomberg-retail-share}{}%
Bloomberg. (2021, November 17). Retail traders slide back below 20\% of
market's total volume.
\url{https://www.bloomberg.com/news/articles/2021-11-17/retail-traders-retreat-as-choppy-markets-challenge-easy-profits}.
Accessed 24 August 2021

\leavevmode\hypertarget{ref-breeden_litzenberger78}{}%
Breeden, D. T., \& Litzenberger, R. H. (1978). Prices of
state-contingent claims implicit in option prices. \emph{The Journal of
Business}, \emph{51}(4), 621--651.
\url{http://www.jstor.org/stable/2352653}

\leavevmode\hypertarget{ref-coins_with_benefits}{}%
Chen, C. Y., \& Vinogradov, D. (2021). Coins with benefits: On
existence, pricing kernel and risk premium of cryptocurrencies.
\emph{Available at SSRN 3959984}.
\url{https://doi.org/http://dx.doi.org/10.2139/ssrn.3864578}

\leavevmode\hypertarget{ref-choi}{}%
Choi, S., \& Shin, J. (2022). Bitcoin: An inflation hedge but not a safe
haven. \emph{Finance Research Letters}, \emph{46}, 102379.
\url{https://doi.org/https://doi.org/10.1016/j.frl.2021.102379}

\leavevmode\hypertarget{ref-oil_spd}{}%
Christoffersen, P., Jacobs, K., \& Pan, X. (. (2021). The State Price
Density Implied by Crude Oil Futures and Option Prices. \emph{The Review
of Financial Studies}, \emph{35}(2), 1064--1103.
\url{https://doi.org/10.1093/rfs/hhab011}

\leavevmode\hypertarget{ref-cryptonews_bitcoin_retail_share}{}%
Coinbase, S., Cryptonews. (2022). Institutions \& retail compete for
bitcoin - whose hands are stronger?
\url{https://cryptonews.com/exclusives/institutions-retail-compete-for-bitcoin-which-is-the-biggest-9695.htm}.
Accessed 22 March 2022

\leavevmode\hypertarget{ref-decentralized_wash_trading}{}%
Cong, L., Li, X., Tang, K., \& Yang, Y. (2019). Crypto wash trading.
\emph{SSRN Electronic Journal}.
\url{https://doi.org/10.2139/ssrn.3530220}

\leavevmode\hypertarget{ref-jackwerth_pk}{}%
Cuesdeanu, H., \& Jackwerth, J. C. (2018). The pricing kernel puzzle:
survey and outlook. \emph{Annals of Finance}, \emph{14}(3), 289--329.
\url{https://doi.org/10.1007/s10436-017-0317-9}

\leavevmode\hypertarget{ref-cross_border_payments}{}%
Deng, Q. (2020). Application analysis on blockchain technology in
cross-border payment. In.
\url{https://doi.org/10.2991/aebmr.k.200306.050}

\leavevmode\hypertarget{ref-deribit-insurance-fund}{}%
Deribit. (2020a). Deribit insurance fund.
\url{https://www.deribit.com/main\#/insurance}. Accessed 4 January 2021

\leavevmode\hypertarget{ref-synthetic-btc-index}{}%
Deribit. (2020b). Deribit btc-usd index.
\url{https://www.deribit.com/main\#/indexes}. Accessed 2 January 2021

\leavevmode\hypertarget{ref-market-maker-obligations}{}%
Deribit. (2020c). Deribit market maker obligations.
\url{https://www.deribit.com/pages/docs/options}. Accessed 2 January
2021

\leavevmode\hypertarget{ref-deribit_retail_piechart}{}%
Deribit. (2022a). Deribit volume share.
\url{https://www.deribit.com/statistics/BTC/market-data}. Accessed 25
July 2022

\leavevmode\hypertarget{ref-deribit_margin}{}%
Deribit. (2022b). Deribit margin requirements.
\url{https://legacy.deribit.com/pages/docs/options}. Accessed 2 May 2022

\leavevmode\hypertarget{ref-deribit_fees}{}%
Deribit. (2022c). Deribit margin requirements.
\url{https://www.deribit.com/kb/fees}. Accessed 2 May 2022

\leavevmode\hypertarget{ref-deribit-rate-limit}{}%
Deribit. (2023). Deribit rate limits.
\url{https://www.deribit.com/kb/deribit-rate-limits}. Accessed 16 April
2021

\leavevmode\hypertarget{ref-vola_premium}{}%
Eraker, B. (2021). The volatility premium. \emph{Quarterly Journal of
Finance (QJF)}, \emph{11}(03), 1--35.
\url{https://EconPapers.repec.org/RePEc:wsi:qjfxxx:v:11:y:2021:i:03:n:s2010139221500142}

\leavevmode\hypertarget{ref-networkeffect}{}%
Gandal, N., \& Halaburda, H. (2016). Can we predict the winner in a
market with network effects? Competition in cryptocurrency market.
\emph{Games}, \emph{7}(3). \url{https://doi.org/10.3390/g7030016}

\leavevmode\hypertarget{ref-Grith09sim}{}%
Grith, M., Härdle, W. K., \& Park, J. (2009). Shape invariant modelling
pricing kernels and risk aversion.
\url{https://doi.org/10.2139/ssrn.2894289}

\leavevmode\hypertarget{ref-hall_1991}{}%
Hall, P. (1991). Edgeworth expansions for nonparametric density
estimators, with applications. \emph{Statistics}, \emph{22}(2),
215--232. \url{https://doi.org/10.1080/02331889108802305}

\leavevmode\hypertarget{ref-hayes}{}%
Hayes, A. S. (2019). Bitcoin price and its marginal cost of production:
Support for a fundamental value. \emph{Applied Economics Letters},
\emph{26}(7), 554--560.
\url{https://doi.org/10.1080/13504851.2018.1488040}

\leavevmode\hypertarget{ref-w2yo}{}%
Härdle, W. K., Okhrin, Y., \& Wang, W. (2014). Uniform Confidence Bands
for Pricing Kernels. \emph{Journal of Financial Econometrics},
\emph{13}(2), 376--413. \url{https://doi.org/10.1093/jjfinec/nbu002}

\leavevmode\hypertarget{ref-pricing_cryptocurrency_options}{}%
Hou, A. J., Wang, W., Chen, C. Y. H., \& Härdle, W. K. (2020). Pricing
Cryptocurrency Options*. \emph{Journal of Financial Econometrics},
\emph{18}(2), 250--279. \url{https://doi.org/10.1093/jjfinec/nbaa006}

\leavevmode\hypertarget{ref-xfg_spd}{}%
Huynh, K., Kervella, P., \& Zheng, J. (2002). \emph{Estimating
state-price densities with nonparametric regression} (SFB 373 Discussion
Papers No. 2002,40). Humboldt University of Berlin, Interdisciplinary
Research Project 373: Quantification; Simulation of Economic Processes.
\url{https://EconPapers.repec.org/RePEc:zbw:sfb373:200240}

\leavevmode\hypertarget{ref-pumpndump}{}%
La Morgia, M., Mei, A., Sassi, F., \& Stefa, J. (2020). Pump and dumps
in the bitcoin era: Real time detection of cryptocurrency market
manipulations. In \emph{2020 29th international conference on computer
communications and networks (icccn)} (pp. 1--9).
\url{https://doi.org/10.1109/ICCCN49398.2020.9209660}

\leavevmode\hypertarget{ref-vola_stylized_facts}{}%
Masset, P. (2011). Volatility stylized facts. \emph{SSRN Electronic
Journal}. \url{https://doi.org/10.2139/ssrn.1804070}

\leavevmode\hypertarget{ref-satoshi2009}{}%
Nakamoto, S. (2009). Bitcoin: A peer-to-peer electronic cash system.
\url{http://www.bitcoin.org/bitcoin.pdf}

\leavevmode\hypertarget{ref-fundamental-theorem-of-asset-pricing}{}%
Pascucci, A., \& Agliardi, R. (2011). PDE and martingale methods in
option pricing. In (pp. 429--495).

\leavevmode\hypertarget{ref-rodgers2019}{}%
Rodgers, T. (2019, April 4). 95 percent of volume could be wash trading
as bitcoin price surges.
\url{https://www.forbes.com/sites/tomrodgers1/2019/04/04/99-of-volume-could-be-wash-trading-as-bitcoin-takes-back-5000/?sh=4640bcbc23d5}.
Accessed 15 April 2023

\leavevmode\hypertarget{ref-Rookley97}{}%
Rookley, C. (1997). Fully exploiting the information content of intra
day option quotes: Applications in option pricing and risk management.

\leavevmode\hypertarget{ref-sar2017}{}%
Sar, M. (2017). Dodd-frank and the spoofing prohibition in commodities
markets.

\leavevmode\hypertarget{ref-deriv_volume}{}%
Skew.com. (2022). Open interest for bitcoin options over time. Skew is a
leading crypto data analytics platform for institutional clients.
\url{https://analytics.skew.com/dashboard/bitcoin-options}. Accessed 25
July 2022

\leavevmode\hypertarget{ref-brazilian_speculation}{}%
Smaniotto, E. N., \& Neto, G. B. (2022). Speculative trading in bitcoin:
A brazilian market evidence. \emph{The Quarterly Review of Economics and
Finance}, \emph{85}, 47--54.
\url{https://doi.org/https://doi.org/10.1016/j.qref.2020.10.024}

\leavevmode\hypertarget{ref-cross_border_stellar}{}%
Stellar Development Foundation. (2022). 2022 research reveals high
awareness and growing cross-border use of cryptocurrency in four key
markets.
\url{https://www.prnewswire.com/news-releases/2022-research-reveals-high-awareness-and-growing-cross-border-use-of-cryptocurrency-in-four-key-markets-301512401.html}.
Accessed 22 July 2022

\leavevmode\hypertarget{ref-nickszabo}{}%
Szabo, N. (2008). Bit gold.
\url{https://unenumerated.blogspot.com/2005/12/bit-gold.html}. Accessed
28 July 2022

\leavevmode\hypertarget{ref-tuccella2021}{}%
Tuccella, J.-N., Nadler, P., \& Șerban, O. (2021). Protecting retail
investors from order book spoofing using a gru-based detection model.

\end{document}
